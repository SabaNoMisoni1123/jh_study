\documentclass{tarticle}

\begin{document}

	\section{論説文(説明文)の読解}

	\subsection{論説文(説明文)とは}

	論説文とは、筆者が主張を伝えることを目的とした文章である。
	読者は筆者の言いたいことを理解することが求められる。
    一方で、文章自体も読者に主張を伝えることを目的としている。
	そのため、読者は論説文のすべてを漠然と読むことではなく、筆者が注目してほしい部分を選んで読むことが必要である。
    言い換えると、論説文を読むとは筆者が注目してほしい部分を見つけることである。

    筆者は文章を書くときに、自分の主張がわかりやすく伝わるための表現を使う。
    大事な部分が大事であると読者がわかる書き方をするのだ。
    筆者が注目してほしい部分を探すとは、論説文中の特定の表現を探すことだとわかる。

    ここまでの話を一旦まとめる。
    \begin{itemize}
        \item 論説文は筆者の主張を伝えるための文章である。
        \item 筆者は自分の主張をわかりやすく伝えるために、ここが大事だとわかる表現を使う。
        \item 論説文の読解とは、筆者の主張が書いている部分を探すことである。
        \item 筆者の主張は特定の表現で書かれているため、筆者の主張を探すとは特定の表現を探すことである。
    \end{itemize}
    さらに付け加えると、論説文の読解問題を解いても論説文を内容を完全に理解することはない
    \footnote{某予備校の国語科講師によると、論説文を読んでみて5割、問題を解いて8割理解できたらいいらしい。
    実際に、私自身も論説文の読解で文章を完全に理解することはない。(それでも8割以上は正解する)}。



	\subsection{読解の手順}

	問題を解く手順としては以下のものを勧める。必ずしもこれに従う必要はない。

	\begin{enumerate}
		\item 問題の文章をすべて読む。このときに、表現から大事であるとわかる部分に線を引いて印をつける。
		\item 設問の文を読んで問題の種類を把握し、解答の骨組みを作る。
		\item 骨組みを作った結果、解答に必要になる内容を本文中の線を引いたところを中心に探す。
		\item 本文から探してきた内容を骨組みに加えることで、解答を完成させる。
	\end{enumerate}

	この手順で想定しているのは記述解答のみである。選択解答の場合は記述解答をするように解答を作成したあとで、
	最も近いものを選ぶようにする。
	そのほかの問題でも、ここに示した手順をマスターすれば問題ないだろう。
	
	
	\section{筆者の主張の発見}
    筆者の主張を探すことが読解であると説明してきた。
    本節では、筆者の主張がどのような表現で書かれているのかについて述べる。

	\subsection{表現でわかる重要さ}
	筆者は読者に対して効果的に主張を伝えるために、その部分が大事であるとわかるような表現を使う。
	代表的なものを挙げると次のようになる。
    すべてBが筆者が読者に伝えたい主張である。

	\begin{description}
		\item[否定] Aではなく、Bである。
		\item[定義] Aとは、Bである。
		\item[添加] Aであることに加えて、Bである。
		\item[強調語句] 「こそ」、「まさに」、「重要な」、「大事な」などのわかりやすい語句による強調。
	\end{description}
	これらの表現がすべてというわけではないが、まずはこの四つを覚えるとよい。
    論説文を読む間にこの表現を見つけたら印をつけるようにする。
    意味を理解できたら一番いいが、わからなくても問題ない。
    印をつけておくことで、後で問題を解くときに役に立つ。

	\subsection{例を利用した主張}
    大事な主張の伝え方は表現だけではない。
    例を利用して主張を読者に伝える手法がある。
	例を挙げるというのはある意味では遠回りであり、またその半面で主張をわかりやすく伝えることにつながる。
	わかりやすく伝わる手段を用いるということは、例が説明する内容はそのまま筆者の伝えたいことであるということである。
	これらをまとめると、文章中の例(具体例)はそれ自体はそこまで必要ないが例が説明する内容は大事であるとわかる。

    次に例を用いて主張を伝える文章を示す。
	\begin{quotation}
		スマートフォンの普及が人とインターネットとの距離を縮めている。
		何かわからないことがあれば「ググる」し、暇になればYouTubeで動画鑑賞をする。
		SNSはインターネットを利用した最たる例であり、皆はそれを通して他者との
		つながりを持ったりもしている。
		インターネットに接続できる端末を持ち歩くようになったことで、日常の生活に
		インターネットのもたらす恩恵があふれるようになったのだ。
	\end{quotation}
	この例文では、二文目、三分目が具体例になっている。そして一文目、四文目はともに具体例が
	何を主張したいのかを示している。どちらも「スマホでインターネットが身近になった」と主張していて、
	具体例が身近になったインターネットの例を示しているのだ。

	この例文から理解してほしいのは、具体例の前後には例が何の主張の手助けをするものなのかが
	示してあるということである。そして、それは筆者の主張に間違いない。
	具体例を発見したらその前後から筆者の主張を探すべきだ。
    大抵は前後の両方に筆者の主張が示してあるが、どちらか一方が欠けている場合がある。


	\subsection{一般論の否定}
	強力に主張を展開したい場合は、初めに一般的な考えを紹介したうえでそれを否定するように自らの主張を
	展開する手法がある。次の例文を読んでみよう。

	\begin{quotation}
		無死一塁では送りバントをするのが定石といわれている。しかし、実際は送りバントによって
		得点できる確率が大幅に上がらない。・・・
	\end{quotation}

	皆が考えることをいちいち文章にはしない。基本的には筆者の主張は一般論とは違う部分がある。
	そのため例文のような表現が散見される。「しかし」という語句から、強調されていると読み取れるが
	それ以上に一般論の否定であるので強調の度合いは大きいことがわかる。






\end{document}
