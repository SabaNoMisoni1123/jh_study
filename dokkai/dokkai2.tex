\documentclass{tarticle}
\usepackage{plext}

\begin{document}
	\section{解答の骨組みの作成}
	実をいうと,文章をほとんど読まずしても解答の骨組みを作ることができる.
	つまり,解答の骨組みとは問題文から作成するのだ.
	これには問題の種類の理解が必要となる.

	\subsection{問題の種類}
	記述解答が求められる問題は大きく二つの種類に分かれる.そして,その二つの他にはない
	\footnote{強く言い切る自信はないが,紹介する二種にまとめられるという理解をしてもらえると助かる.}.

	\begin{description}
		\item[内容説明問題] 「傍線部とはどういうことか.」,「傍線部で筆者は何を主張しているか.」,「傍線部は何を意味しているのか.」
			というような問題である.問題の意図は傍線部をわかりやすくせよということである.
		\item[理由説明問題] 「傍線部であるのはなぜか.,「傍線部と筆者が考えるのはどうしてか.」,「なぜ傍線部といえるか.」
			というような問題である.問題の意図は文章中の飛躍を補うということである.
	\end{description}


	これらの二種にはそれぞれ定式化された骨組みが存在する.
	それらの理解が文章の読み方を決定する.

	\subsection{内容説明問題}
	内容説明問題とは,傍線部言い換え問題である.
	傍線部に書かれている事柄をわかりやすくすべて言い換える
	ことが求められている
	\footnote{傍線部を言い換えただけでは足りない.傍線部を含む一文を言い換えて初めて得点されると考えたほうが良い}.
	大抵の場合,傍線部中には「この」や「そのような」といった指示語があるので,
	それらは正しく指示されている内容を補えばいい.
	具体例を示し理解を深める.

	\begin{quotation}
		それによって生まれるインターネット利用の変化は, \kasen{若者だけでなく}皆に影響を与える.

		[問題] 傍線部とはどういうことか.
	\end{quotation}

	傍線部を含む一文すべてを言い換えれば答えとなる.解答の骨組みは次の通りとなる.

	\begin{quotation}
		(「それによって」の指す内容)によってインターネット利用がAからBへと変わることは,若者だけでなく
		Cにも(D:影響の内容)という影響を与えるということ.
	\end{quotation}

	あとは「それによって」が指す内容,A,B,C,Dの内容を本文中から探して骨組みに加えることで答えとなる.
	ここでは変化の言い換えに何から何までということを加え,さらに影響が具体的に何なのかということを加えている.
	これがどこまでできるかが満点の解答との分かれ道となる.
	ある言葉を説明するときに,どれだけの要素があれば十分なのかということを常々考える必要がある.

	\subsection{理由説明問題}
	理由説明問題とは,内容説明問題のように傍線部の言い換えをしたうえで,論の飛躍を補う問題である.論の飛躍は次のようなものである.

	\begin{quotation}
		イチローはすごい.(なぜすごいのか,野球を知らない人,イチローを知らない人にはわからない.)

		大リーグでも活躍したイチローは,10年連続200本安打を達成したのですごい.
	\end{quotation}

	AはBだ.ということはなかなか伝わるものではない.AはCだからBだ.というように理由を補う必要があるのだ.
	これを問うのが理由説明問題である.ここでは具体例を示さないが,基本的には傍線部言い換えをしっかり行った
	上で,理由を加えればよい.

	理由のみを探せばよいように感じるが,問題が難しくなるにつれて傍線部を含む一文の理解が十分でないと
	誤った理由を探し出してしまう場合がある.解答に必要がなくても,必ず傍線部を含む一文に目を通さなければならない.


	% \subsection{表現問題(高校生向けの内容)}
	% 記述解答ではない問題形式であるが,筆者の表現についてその意図を答えさせる問題がセンター試験に出てくる.



\end{document}
