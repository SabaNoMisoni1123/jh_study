\documentclass[dvipdfmx]{beamer}

\usepackage{graphics}
\usepackage{amsmath}
\usepackage{amssymb}
\usepackage{ascmac}
\usepackage{txfonts}
\usepackage{bm}
\usepackage{docmute}
\usepackage{tikz}
\usetikzlibrary{calc}
\usetikzlibrary{intersections}

\usetheme{Madrid}
\usefonttheme{professionalfonts}

\title{コインの確率}
\author{近藤綜太}

\begin{document}
\maketitle
\begin{frame}{問題}\centering
	10枚のコインを同時に投げた時,\\表になるコインの枚数が2枚上となる確率を求めよ.
	
\end{frame}
\begin{frame}{方針}
	コインの表の枚数を2枚,3枚,4枚,,,と考えるのは
	あまりにも面倒で遠回り.
	そこで,表のコインが0枚の時と,1枚の時を考えて
	その確率を1から引く.
	\[(\text{表が1枚以下の確率})+(\text{表が2枚以上の確率})=1\]
	起きそうなこと全ての確率を足せば,1になるという
	確率の重要なルールを使う.

	求めたい方ではない方の確率を余事象の確率という.
	\[(\text{求めたい確率})=1 - (\text{余事象の確率})\]
\end{frame}

\begin{frame}{確率計算でのルール}

	確率の計算をする上で重要なルールとして
	\begin{center}
		「区別のないものには名前をつけて区別する.」
	\end{center}
	というものがある.
	今回の問題では,10枚のコインが区別のないものである.
	これらにはコイン1からコイン10までの名前をつけておく.

	というのも,例えば1枚が表になる確率を考える時,
	10枚のうち1枚だから $1/2^{10}$では間違いなのである.
	コイン1だけ表の確率 $1/2^{10}$,コイン2だけが表の確率 $1/2^{10}$,,,
	を全て足し合わせて,
	\[(\text{表が1枚の確率})= \frac{10}{1024} \quad (1024 = 2^{10})\]
	としければならない.

	他にもくじ引きの当たりくじ,ハズレくじにも名前をつける方が良い場合がある.

\end{frame}

\begin{frame}{解答}
	余事象を考える.表の枚数が0の時の確率を $P_0$,1枚の時の確率を
	$P_1$とする.これらは以下で計算される.
	\begin{align*}
		P_0 &= \frac{1}{2^{10}} = \frac{1}{1024}\\
		P_1 &= \frac{10}{2^{10}}= \frac{10}{1024}
	\end{align*}
	求める確率 $P$は次で計算され,答えとなる.
	\[P = 1 - P_0 - P_1 = \frac{1013}{1024}\]

\end{frame}
	
\end{document}
