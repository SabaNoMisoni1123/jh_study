\documentclass[dvipdfmx]{beamer}

\usepackage{graphics}
\usepackage{amsmath}
\usepackage{amssymb}
\usepackage{ascmac}
\usepackage{txfonts}
\usepackage{bm}
\usepackage{docmute}
\usepackage{tikz}
\usetikzlibrary{calc}
\usetikzlibrary{intersections}

\usetheme{Madrid}
\usefonttheme{professionalfonts}

\title{因数分解}
\author{近藤綜太}
\begin{document}
\maketitle
	\begin{frame}{因数分解の手順}
		因数分解は以下の手順を辛抱強く続ける.
		\begin{enumerate}
			\item 式中のいくつかの項に対して共通因数のくくりだし
			\item 文字の置き換え(絶対ではない)
			\item 公式の利用
		\end{enumerate}
		いきなり公式が使えることはまずない.共通因数のくくりだしを
		繰り返しながら,公式が使える形や同じ形が出てくるところを探す.
	\end{frame}

% 問題のセクション

	\begin{frame}{問題1}
		$(a^2+b^2-c^2)^2$を展開すると(1)であるから,
		$a^4+b^4+c^4-2a^2b^2-2b^2c^2-2c^2a^2$を因数分解すると
		(2)となる.(灘高)
	\end{frame}

	\begin{frame}{問題2}
		以下の式を因数分解せよ.
		\begin{align*}
			&ax^2+7ax+12a & &3ax^2-18ax+24a\\
			&6x^2-6 & &25ax^2-16a\\
			&3x^2-30x+75 & &18ax^2+24ax+8a\\
			&(5x-1)^2 - y^2 & &(x+y)^2+4(x+y)+4\\
			&(x+a)^2-(y+5)^2 & &(x+3)^2-2(x+3)-63
		\end{align*}
	\end{frame}


% 解答のセクション

\begin{frame}{式の展開}
	式の展開はカッコの中から1つずつ項を取り出してペアを作る作業.
	\[(a+b)(c+d)\]
	から,前のカッコと後ろのカッコから1つずつ取り出すことでペアを作る.
	この時できるペアは $(a, c), (a, d), (b, c), (b, d)$の4つ.
	ペアを全て掛け算して足すので
	\[(a+b)(c+d)=ac+ad+bc+bd\]
	となる.全てのペアを忘れずに書けば式の展開は間違いはない.
\end{frame}

\begin{frame}{解答1}
	(1)は式の展開.丁寧に計算すれば以下の式となる.
	\begin{align*}
		(a^2+b^2-c^2)^2 &= a^4 + b^4 + c^4 +2(a^2b^2 - b^2c^2 -c^2a^2)\\
						&= a^4 + b^4 + c^4 +2a^2b^2 -2 b^2c^2 -2c^2a^2
	\end{align*}
	これと問題文の式を比較すると $4a^2b^2$だけ問題文の式が小さいことが
	わかる.ここから次の計算によって因数分解がわかる.
	\begin{align*}
		&a^4+b^4+c^4-2a^2b^2-2b^2c^2-2c^2a^2 \\
		&=(a^2+b^2-c^2)^2 - 4a^2b^2\\
		&=\{(a^2+b^2-c^2)+2ab\}\{(a^2+b^2-c^2)-2ab\}\\
		&=\{(a+b)^2-c^2\}\{(a-b)^2-c^2\}\\
		&=(a+b+c)(a+b-c)(a-b+c)(a-b-c)
	\end{align*}

\end{frame}

% \begin{frame}{問題1(別解)}
%     誘導がなくても答えにたどり着きたい.
% \end{frame}

\begin{frame}{因数分解の基本}
	因数分解の基本は次の3つ
	\begin{enumerate}
		\item $a^2-b^2=(a+b)(a-b)$
		\item $x^2+(a+b)x+ab=(x+a)(x+b)$
		\item たすきがけ
	\end{enumerate}
	1つ目は和と差の積と表現される公式.
	問題式から見えるかどうかなので俗に視力検査
	\footnote{見えたら間違うことはない.視力検査と同じ.}という.
	2つ目は式の展開公式を逆にしたもの. $a, b$を見つければ
	因数分解ができる.見つけ方としては,まず $ab$に注目する.
	3つ目はたすきがけ.因数分解で一番難しいもので,
	定期試験に出ることはそんなにない.入試にはバンバン出る(かな).
\end{frame}
\begin{frame}{たすきがけ}
	\[(ax+b)(cx+d)=abx^2+(ad+bc)x+bd\]
	この展開の式を参考にして $a, b, c, d$を探す.
	探し方としては $ac, bd$に注目して $(a, c), (b, d)$
	の候補をたくさん用意しいて,全てに
	$(ad+bc)$を計算する.

	2番目,3番目の因数分解に近道はなく,候補となる
	$a, b$などをまとめたら,全て計算して $x$の係数を確認する.
	
	たすきがけは特に面倒なので,共通因数のくくりだしは必ず
	前にやっておく.

\end{frame}


\begin{frame}{解答2-1}
	\begin{align*}
		ax^2+7ax+12a &= a(x^2+7x+12)\\
					 &= a(x+4)(x+3)
	\end{align*}
	\begin{align*}
		3ax^2-18ax+24a &=3a(x^2-6x+8)\\
					   &=3a(x-4)(x-2)
	\end{align*}
	\begin{align*}
		6x^2-6 &= 6(x^2-1)\\
			   &=6(x-1)(x+1)
	\end{align*}
	\begin{align*}
		25ax^2-16a &= a(25x^2-16)\\
				   &= a(5x+4)(5x-4)
	\end{align*}
	% 共通因数のくくりだしを忘れなければ簡単に計算ができる.
	% 因数分解の公式が身についていない場合は,
	% 式の展開の部分に戻ってたくさん問題を解くと
	% いいと思う.

\end{frame}

\begin{frame}{解答2-2}
	\begin{align*}
		3x^2-30x+75&= 3(x^2-10x+25)\\
				   &= 3(x-5)^2
	\end{align*}
	\begin{align*}
		18ax^2+24ax+8a&= 2a(9x^2+12x+4)\\
					  &= 2a(3x+2)^2
	\end{align*}

	2乗の展開公式を覚えておくと
	すぐに計算できる.
	何かを2乗したものが出てきたら,疑わないといけない.
	\begin{align*}
		(x+a)^2 &= x^2+2ax+a^2\\
		(ax+b)^2 &= a^2x^2 +2abx+b^2
	\end{align*}
	
\end{frame}

\begin{frame}{解答2-3}
	\begin{align*}
		(5x-1)^2 - y^2&= \{(5x-1)+y\}\{(5x-1)-y\}\\
					  &= (5x+y-1)(5x-y-1)
	\end{align*}
	\begin{align*}
		(x+y)^2+4(x+y)+4&=X^2 +4X +4\qquad (X=x+y)\\
						&= (X+2)^2\\
						&= (x+y+2)^2
	\end{align*}
	\begin{align*}
		(x+a)^2-(y+5)^2&= \{(x+a)+(y+5)\}\{(x+a)-(y+5)\}\\
					   &=(x+y+a+5)(x-y+a-5)
	\end{align*}
	\begin{align*}
		(x+3)^2-2(x+3)-63 &= X^2-2X-63\qquad(X=x+3)\\
						  &= (X-9)(X+7)\\
						  &= (X-6)(x+10)
	\end{align*}
	
\end{frame}

\begin{frame}{所感}
	因数分解では式を見て公式を見つけられるか,くくりだすのに
	ちょうどいい共通因数を見つけられるのかにかかっている.
	これはこうするというような公式があるわけではないので,
	たくさん問題を解いて覚えるしかない.
\end{frame}
\end{document}
