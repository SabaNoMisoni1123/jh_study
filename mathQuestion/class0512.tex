\documentclass[dvipdfmx]{beamer}
\usepackage{graphics}
\usepackage{amsmath}
\usepackage{amssymb}
\usepackage{amsthm}
\usepackage{ascmac}
\usepackage{mathtools}
\usepackage{bm}
\usepackage{url}
\usepackage{txfonts}
\usepackage{color}
\usepackage{docmute}
\usepackage{tikz}
\usetikzlibrary{calc}
\usetikzlibrary{intersections}

\usetheme{Madrid}
\usefonttheme{professionalfonts}

\title{連立方程式を利用する文章題}
\author{近藤 綜太}

\begin{document}
	\maketitle
	\begin{frame}{問題}
		\begin{enumerate}
			\item 
				A,Bの2つの品物を仕入れた.
				AとBの仕入れ値の比は45:44
				であった.
				今,Aに20\%,Bに25\%の利益を
				見込んで定価をつけると
				Aの方が200円安くなった.
				AとBの仕入れ値は
				それぞれ何円だったか
				求めなさい.
			\item 
				毎日一定の水量が
				放水され,
				一定の水量が
				注入されている貯水池に,
				現在,水が180万L
				残っている.
				現在の放水量では
				60日間で貯水池は
				空になってしまい,
				放水量を現在の25\%
				増しにしたとすると,
				36日間で
				貯水池は空に
				なってしまうという.
				現在の1日あたりの
				放水量と,
				1日あたりの
				流入量をそれぞれ
				求めなさい.
		\end{enumerate}
		
	\end{frame}

	\begin{frame}{解答1}
		1問目は
		AとBの仕入れ値を求める問題である.
		なので,AとBの仕入れ値をそれぞれ $x, y$として
		連立方程式を利用することを考える.

		%\pause

		問題文を確認するとそこには仕入れ値に関して2つの
		事柄が書いてあることがわかる.
		\begin{itemize}
			\item 仕入れ値の比が45:44
			\item 定価をつけると200円の差
		\end{itemize}
		この2つの事柄を式として表現できれば,
		あとは連立方程式を解くことで答えを得ることができる.
	\end{frame}

	\begin{frame}{解答2}
		仕入れ値についての式を立てる.AとBの仕入れ値の比は45:44より
		次の式である.
		\[x:y = 45:44\]
		比の式のままでは計算しにくいので次式に変形する.
		\[44x=45y\, \Leftrightarrow \, 44x-45y=0\]

		%\pause

		次は定価についてである.20\%の利益を見込むとは
		仕入れ値に1.2をかけること,25\%に対しても同じように1.25を
		かけることである.よってAの定価は $1.2x$,Bは $1.25y$である.
		Aの定価がBの定価より200円安いので次の式になる.
		\[1.2x-1.25y=-200\]
	\end{frame}

	\begin{frame}{解答3}
		ここまでで次の連立方程式を得る.
		\[ \begin{cases}{}
			44x-45y=0&(1)\\
			1.2x-1.25y=-200&(2)
		\end{cases}\]

		%\pause

		$(1)-36\times(2)$より
		\begin{gather*}
			0.8x=7200\\
			x=9000
		\end{gather*}
		(1)より
		\[y= \frac{44}{45}x=8800\]

		%\pause

		以上より答え
		\begin{center}
			Aの仕入れ値9000円,Bの仕入れ値8800円.
		\end{center}

	\end{frame}

	\begin{frame}{解答4}
		2問目は
		1日の放水量と流入量の2つを求める問題である.
		放水量を $x\, \text{万}\mathrm{L}/\text{日}$,
		流入量を $y\, \text{万}\mathrm{L}/\text{日}$とする.

		%\pause

		この状態での
		1日の水量の変化は $-x+y\, \text{万}\mathrm{L}/\text{日}$である.
		60日つづけば180万Lがなくなる.
		なので以下の関係を得る.
		\[60(-x+y)=-180\]
		変形することで連立方程式の1つの式を得る.
		\[-x+y=-3\]

		%\pause

		放水量を25\%増しにするとは $x\to1.25x$と考えることである.
		この状態での1日の水量の変化は
		$-1.25x+y\, \text{万}\mathrm{L}/\text{日}$である.
		36日で180万Lがなくなるので次の関係を得る.
		\[36(-1.25x+y)=-180\]
		変形することで連立方程式のもう1つの式を得る.
		\[-1.25x+y=-5\]
	\end{frame}


	\begin{frame}{解答5}
		\[ \begin{cases}{}
			-x+y=-3&(1)\\
			-1.25x+y=-5&(2)
		\end{cases}\]

		%\pause

		$(1)-(2)$より,
		\begin{gather*}
			0.25x = 2\\
			\therefore x=8
		\end{gather*}
		(1)より
		\[y=-3+x=5\]

		%\pause
		
		以上より答え,
		\begin{center}
			流出量8万L,流入量5万L.
		\end{center}
		
	\end{frame}
\end{document}
