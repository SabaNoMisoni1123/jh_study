\documentclass[dvipdfmx]{jsarticle}
\usepackage{graphics}
\usepackage{amsmath}
\usepackage{amssymb}
\usepackage{amsthm}
\usepackage{ascmac}
\usepackage{mathtools}
\usepackage{bm}
\usepackage{url}
\usepackage{txfonts}
\usepackage{color}
\usepackage{docmute}
\usepackage{tikz}
\usetikzlibrary{calc}
\usetikzlibrary{intersections}

\begin{document}
	\section*{現在完了}
	\subsection*{先週}
	現在完了は幅を持った時間を表す文法.外人にとってはそれだけで
	充分だけど,日本語に直そうと思うと次の3種類に分かれる.
	\begin{itemize}
		\item 継続
		\item 経験
		\item 完了
	\end{itemize}
	
	\subsection*{59}

	I have not visited my uncle since 2008.

	私は叔父を2008年から訪れていない.

	I have not cleaned my room for two week.

	私は二週間部屋を掃除していない.

	She has not been at home all this morning.

	午前中の間,彼女は家にいなかった.
	
	\subsection*{60}
	(1) Have you been busy since this morning?

	Yes, I have.

	(2) Has she been in the hospital since Monday?

	Yes, she has.

	(3) Has he been in the library for three hours?

	No, he hasn't.

	(4) How long have you been friends with her?

	For two years.

	\subsection*{70}
	(1) Have you ever read an English story?

	(2) Have you ever been to Australia?

	(3) Have you ever heard of this animal?

	(4) Has your sister worked at this shop before?

	(5) How many times have you ever been to foreign countries?
\end{document}
